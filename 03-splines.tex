\chapter{\textit{SPLINES} CÚBICAS} \label{cha:splines}

As \textit{splines} são funções definidas por partes que permitem a construção de curvas suaves a partir de um conjunto de pontos de controle. Elas são amplamente utilizadas em contextos onde é necessário representar formas complexas de maneira precisa e contínua, como na computação gráfica, modelagem geométrica e análise de dados. Dentre os diversos tipos, as \textit{splines} cúbicas se destacam por proporcionarem um bom equilíbrio entre flexibilidade e suavidade, garantindo continuidade até a segunda derivada e, consequentemente, uma transição harmoniosa entre os segmentos da curva.

Na área de biometria facial, a representação precisa das características do rosto — como contornos dos olhos, nariz e lábios — é essencial para o desempenho de sistemas de reconhecimento. Utilizar \textit{splines} para descrever essas formas permite capturar variações sutis da geometria facial com suavidade e fidelidade, o que é especialmente importante na comparação entre diferentes imagens. Além disso, a estrutura matemática das \textit{splines} facilita tanto a manipulação quanto a análise das curvas faciais, tornando-as uma ferramenta eficiente e robusta nesse tipo de aplicação.

Neste capítulo, será abordado as \textit{splines} cúbicas, suas propriedades e aplicações. A sua construção envolve a definição de um conjunto de pontos de controle e a determinação dos coeficientes que definem os polinômios cúbicos entre esses pontos. A suavidade da curva é garantida pela imposição de condições de continuidade e derivabilidade, resultando em uma representação suave e flexível.

\section{Definição de \textit{Splines} Cúbicas}
\label{sec:definicao-splines-cubicas}
Uma \textit{spline} cúbica é uma função definida por partes, onde cada parte é um polinômio cúbico. Seguimos então para os polinômios de grau 3, que são especificados na forma de Hermite:
\begin{equation}
    p(t) = c_0 + c_1t + c_2t^2 + c_3t^3,
\end{equation}
onde cada $c_i$ é um vetor no $\mathbb{R}^2$.

A função $p$ representa uma curva parametrizada cuja variável independente é \( t \in [0,1] \). Essa função descreve a posição ao longo da curva no plano bidimensional, sendo essa posição o elemento de interesse ao desenharmos a curva, já que o processo de plotagem envolve a determinação de pontos com coordenadas $(x, y)$. As curvas de Hermite são definidas por equações paramétricas, ou seja, ambas as componentes $x$ e  $y$ são expressas como funções da variável $t$. Para ilustrar esse conceito, iniciamos definindo $x$ em função de $t$:

\begin{equation}
    x(t) = c_0 + c_1t + c_2t^2 + c_3t^3
\end{equation}

A função \( y(t) \) será expressa de forma análoga à função \( x(t) \), embora possivelmente com coeficientes diferentes. A variável \( t \) representa o parâmetro que indica o progresso ao longo da curva, normalizado no intervalo \( [0,1] \), sendo \( t = 0 \) correspondente ao início da curva e \( t = 1 \) ao seu final.

Expressar uma curva cúbica na forma de Hermite consiste em fornecer quatro valores fundamentais — conhecidos como valores de Hermite — que contêm todas as informações necessárias para descrever completamente a curva. Esses valores também permitem calcular os coeficientes da variável \( t \) na equação polinomial cúbica padrão, resultando assim na expressão completa da curva como um polinômio.

A principal motivação para utilizar a forma de Hermite é a possibilidade de determinar os coeficientes do polinômio diretamente a partir das informações das posições dos pontos inicial e final da curva, bem como das tangentes nesses pontos:


\begin{equation}
    p_0 = p(0),
\end{equation}
posição do ponto inicial da curva.

\begin{equation}
    p_1 = p(1),
\end{equation}
posição do ponto final da curva.

\begin{equation}
    v_0 = v(0),
\end{equation}
a taxa de variação em $p_0$.

\begin{equation}
    v_1 = v(1),
\end{equation}
a taxa de variação em $p_1$.

Agora que sabemos como se dão os valores de Hermite, podemos resolvê-los através do seguinte sistema de equações:

\begin{align}
    p(0) &= c_0 + c_1(0) + c_2(0)^2 + c_3(0)^3 = c_0 \\
    p(1) &= c_0 + c_1 + c_2 + c_3
\end{align}


Para encontrar $v_0$ e $v_1$ é apenas calcular a derivada da posição. Portanto:

\begin{align}
    v(t) &= p'(t) = c_1 + 2c_2t + 3c_3t^2\\
    v(0) &= c_1\\
    v(1) &= c_1 + 2c_2 + 3c_3
\end{align}

Então é possível obter a fórmula:

\begin{align}
    c_0 &= p_0\\
    c_1 &= v_0\\
    c_2 &= -3p_0 - 2v_0 - v_1 + 3p_1\\
    c_3 &= 2p_0 + v_0 + v_1 - 2p_1
\end{align}

Escrevendo de forma matricial, temos:

\begin{equation}
    p(t) =  
        \left(
        \begin{array}{rrrr}
            1 & t & t^2 & t^3
        \end{array}\right)
        \left(
        \begin{array}{rrrr}
            1 & 0 & 0 & 0 \\
            0 & 0 & 1 & 0 \\
            -3 & 3 & -2 & -1 \\
                2 & -2 & 1 & 1
        \end{array}\right)
    \left(
        \begin{array}{r}
            p_0 \\
            p_1 \\
            v_0 \\
                v_1
        \end{array}\right)
 \end{equation}

 \subsection{Criação da \textit{Spline}}

 Supondo que existam 4 pontos e é desejado criar uma \textit{spline} que passe por todos eles. Para isso, é necessário criar 3 segmentos de curvas cúbicas, cada uma passando por dois pontos consecutivos. Assim, a \textit{spline} será composta por 3 segmentos cúbicos, cada um definido por 4 pontos de controle. \cite{CatmullRom}

 A primeira curva irá de $p_0$ a $p_1$, a segunda de $p_1$ a $p_2$ e a final de $p2$ a $p_3$. Então aplicaremos a fórmula que derivamos acima para $p_t$, mas em 3 variantes diferentes, cada variante com um par diferente de pontos finais. A fórmula da primeira curva tera pontos finais $p_0$ e $p_1$, a segunda em $p_1$ e $p_2$ e a terceira em $p_2$ e $p_3$, como é possível ver na \autoref{fig:spline_tangentes}. 
 
 Para as tangentes, Edwin Catmull e Raphael Rom descobriram que uma forma de fazer isso é subtrair o ponto anterior do ponto seguinte. Então para calcular a tangente para o ponto $p_2$ é só fazer ($p_3 - p_1$). Também é comum dividir por 2, pois tende a gerar curvas esteticamente agradáveis.

\begin{figure}[h!]
    \caption{CURVAS CÚBICAS DE CATMULL-ROM.}
    \centering
    \includegraphics[width=0.8\textwidth]{fig/s1.jpg}
    \legend{Spline cúbica com 4 pontos de controle}
    \vspace{0.5em} % Espaçamento vertical entre as imagens
    \includegraphics[width=0.8\textwidth]{fig/s2.jpg}
    \legend{Tangentes coincidentes nas conexões das cúbicas em cada ponto.}
    \label{fig:spline_tangentes}
\end{figure}

Logo é possível reescrever a equação em termos de 4 pontos $p_0$ a $p_3$.

Outra complicação que surge é que se precisar usar o ponto anterior e o próximo para calcular a tangente, surge a dúvida de qual será o ponto anterior de $p_i$ e o próximo ponto de $p_f$. Uma solução para isso é adicionar pontos fantasmas que não fazem parte da curva.


Então ficamos com:
\begin{equation}
    p(t) =  
        \left(
        \begin{array}{rrrr}
            1 & t & t^2 & t^3
        \end{array}
        \right)
        \left(
        \begin{array}{rrrr}
            1 & 0 & 0 & 0 \\
            0 & 0 & 1 & 0 \\
            -3 & 3 & -2 & -1 \\
                2 & -2 & 1 & 1
        \end{array}
        \right)
    \left(
        \begin{array}{r}
            p_1 \\
            p_2 \\
            \frac{p_2 - p_0}{2\tau} \\
                \frac{p_3 - p_1}{2\tau}
        \end{array}
        \right)
\end{equation}
, onde $\tau$ é um fator de escala que pode ser ajustado para controlar a suavidade da curva. O valor de $\tau$ pode ser definido com base na distância entre os pontos de controle ou em outras considerações geométricas.

Com isso é possível chegar no formato de Catmull-Rom:

\begin{equation}
    p(t) = \frac{1}{2}
       \left(
 	\begin{array}{rrrr}
 		1 & t & t^2 & t^3
 	\end{array}
    \right)
    \left(
 	\begin{array}{rrrr}
 		0 & 2 & 0 & 0 \\
 	    -\tau & 0 & \tau & 0 \\
 		2\tau & \tau-6 & -2(\tau-3) & -\tau \\
            -\tau & 4-\tau & \tau-4 & \tau
 	\end{array}
    \right)
   \left(
 	\begin{array}{r}
 		p_0 \\
 	    p_1 \\
 		p_2 \\
            p_3
 	\end{array}
    \right)
\end{equation}

Nas seguintes figuras é possível observar que quanto menor a tensão $\tau$, menos suave fica a curva:

\begin{figure}[h!]
    \caption{TENSÃO DA CURVA.}
    \centering
    \begin{minipage}[b]{0.45\textwidth}
        \centering
        \includegraphics[width=1\linewidth]{fig/cat_rom_t1.png}
        \legend{$\tau$ = 1}
        \label{fig:tesao1}
    \end{minipage}
    \hfill
    \begin{minipage}[b]{0.45\textwidth}
        \centering
        \includegraphics[width=1\linewidth]{fig/cat_rom_t05.png}
        \legend{$\tau$ = 0.5}
        \label{fig:tensao05}
    \end{minipage}

    \vspace{1cm}

    \begin{minipage}[b]{0.45\textwidth}
        \centering
        \includegraphics[width=1\linewidth]{fig/cat_rom_t025.png}
        \legend{$\tau$ = 0.25}
        \label{fig:tensao025}
    \end{minipage}
    \label{fig:tensao}
\end{figure}

\subsection{Resultados}

Foi implementado um algoritmo autoral que gera uma \textit{spline} cúbica a partir de um conjunto de pontos de controle. O algoritmo utiliza a fórmula de Catmull-Rom para calcular os coeficientes da curva, garantindo suavidade e continuidade. 

A seguir, são apresentados os resultados obtidos em conjuntos de pontos de controle que foram extraídos da imagem pelo processo explicado no Capítulo \ref{cha:processamento-imagem}.

\begin{figure}[h!]
    \caption{SPLINES NOS PONTOS EXTRAÍDOS.}
    \centering
    \begin{minipage}[b]{0.45\textwidth}
        \centering
        \includegraphics[width=1\linewidth]{fig/07_sorted_points_right_eye.png}
        \legend{Reordenação dos pontos.}
        \label{fig:reordenacao}
    \end{minipage}
    \hfill
    \begin{minipage}[b]{0.45\textwidth}
        \centering
        \includegraphics[width=1\linewidth]{fig/08_spline_plot_right_eye.png}
        \legend{Splines.}
        \label{fig:spline}
    \end{minipage}
\end{figure}





% \begin{algorithm}[H]
%     \renewcommand{\algorithmcfname}{Algoritmo}%
%     \SetAlgoLined
%     \eSe{$n\leq r$}{
%         Calcule a fatoração $\vetor{M}=\vetor{LU}$ por outro método\;
%     }{
%         Calcule a fatoração $\left[\begin{array}{cc} \vetor{A}\\\vetor{C}\end{array}\right]=\left[\begin{array}{cc} \vetor{L_{11}}\\\vetor{L_{21}}\end{array}\right]\vetor{U_{11}}$\;
%         Calcule $\vetor{U_{12}}=\vetor{L_{11}^{-1}}\vetor{B}$ pela solução do sistema linear $\vetor{L_{11}}\vetor{U_{12}}=\vetor{B}$\;
%         $\vetor{\tilde{D}} \gets \vetor{D}-\vetor{L_{21}}\vetor{U_{12}}$\;
%         Calcule a fatoração $\vetor{\tilde{D}}=\vetor{L_{22}}\vetor{U_{22}}$ recursivamente\;
%         Retorne $\vetor{L}=\left[\begin{array}{cc} \vetor{L_{11}} & \vetor{0}\\\vetor{L_{21}} & \vetor{L_{22}}\end{array}\right]$ e $\vetor{U}=\left[\begin{array}{cc} \vetor{U_{11}} & \vetor{U_{12}}\\\vetor{0} & \vetor{U_{22}}\end{array}\right]$\;
%     }
% \end{algorithm}

% \begin{algorithm}
%     \caption{Fatoração QR magra de uma matriz $\vetor{A}\in\R^{m\times n}$ usando Gram-Schmidt.}
%     \SetAlgoLined
%     \Para{$k$ de $1$ até $n$}{
%         $\vetor{v_k} \gets \vetor{A}[:,k]$\;
%         \Para{$i$ de $1$ até $k-1$}{
%             $\vetor{R}[i,k] \gets \vetor{Q}[:,i]^{\vetor{T}}\vetor{A}[:,k]$\;
%             $\vetor{v_k} \gets \vetor{v_k}-\vetor{R}[i,k]\vetor{Q}[:,i]$\;
%         }
%         $\vetor{R}[k,k]\gets\left\|\vetor{v_k}\right\|_2$\;
%         $\vetor{Q}[:,k]\gets\dfrac{1}{\vetor{R}[k,k]}\vetor{v_k}$\;
%     }
%     \label{alg:QRmagra}
% \end{algorithm}