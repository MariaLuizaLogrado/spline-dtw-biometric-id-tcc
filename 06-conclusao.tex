\chapter{CONCLUSÃO} \label{cha:conclusao}

O objetivo central deste trabalho foi investigar a viabilidade do uso de \textit{splines} cúbicas como método de representação de características faciais em sistemas biométricos, buscando unir a expressividade matemática das curvas suaves com a necessidade de precisão e eficiência nos processos de identificação de indivíduos. Diferentemente de abordagens tradicionais baseadas em \textit{pixels}, essa proposta parte da premissa de que a forma das feições faciais contém informações ricas e discriminativas que podem ser representadas de forma mais compacta, interpretável e flexível através de curvas. A metodologia também buscou avaliar o comportamento dessa representação ao ser combinada com uma técnica de comparação temporal não-linear, o \textit{Dynamic Time Warping}, para verificação da similaridade entre diferentes amostras faciais.

Para que essa abordagem fosse possível, foi necessário desenvolver uma etapa robusta de extração automática de pontos. A \textit{pipeline} proposta iniciou-se com a detecção de rostos utilizando classificadores Haar Cascade, seguida pela aplicação do algoritmo de Canny para realçar contornos faciais relevantes, como os olhos, nariz e boca. Esses contornos, ainda com excesso de pontos, foram então refinados por meio de técnicas de processamento baseadas em grafos: utilizando componentes conexos, árvores geradoras mínimas e poda de nós, obteve-se uma versão reduzida e estruturada dos pontos. Essa etapa foi fundamental para garantir que a modelagem por \textit{splines} posteriormente ocorresse de forma eficiente e significativa, preservando os principais traços geométricos do rosto.

A modelagem com \textit{splines} cúbicas foi escolhida pela sua capacidade de representar curvas suaves que passam por um conjunto de pontos com continuidade e derivadas bem definidas. As \textit{splines} atuaram como intermediárias entre a imagem e a abstração matemática das formas faciais, convertendo conjuntos discretos de pontos em curvas contínuas que descrevem com precisão a geometria das feições. Além disso, a representação por \textit{splines} facilita tanto a visualização quanto a análise matemática, o que a torna vantajosa frente a representações puramente estatísticas ou baseadas em aprendizado de máquina de ``caixa-preta''.



Para comparar diferentes curvas, foi utilizado o algoritmo DTW, uma técnica que mede a similaridade entre sequências que podem variar em comprimento ou sofrer pequenas distorções espaciais. O método foi aplicado diretamente sobre os pontos das \textit{splines}, buscando identificar o alinhamento ótimo entre duas formas faciais distintas. Essa abordagem permitiu lidar com variações naturais entre amostras de um mesmo indivíduo -- como leve inclinação do rosto -- sem comprometer a acurácia da comparação. Assim, o DTW mostrou-se especialmente útil em contextos onde as formas das \textit{splines} preservam a identidade, mas podem diferir ligeiramente em suas parametrizações.


Os resultados experimentais demonstraram que a proposta é viável e promissora. A representação com \textit{splines} cúbicas possibilitou capturar de forma eficiente os principais contornos das feições faciais, mesmo após a redução dos pontos extraídos. Além disso, o uso do DTW como medida de similaridade entre curvas permitiu diferenciar indivíduos com boa acurácia, mesmo em um cenário com variações naturais entre capturas faciais. Embora o trabalho tenha utilizado uma base de dados limitada, os testes iniciais indicam que a combinação entre representação geométrica e alinhamento temporal pode oferecer uma solução eficiente e interpretável para aplicações biométricas.

Além dos avanços apresentados, há diversas possibilidades de aprimoramento que podem ser exploradas em trabalhos futuros. Etapas de pré-processamento adicionais, como a rotação e o alinhamento dos rostos com base em pontos de referência (por exemplo, olhos e boca), são práticas comuns na literatura de reconhecimento facial e podem contribuir para maior robustez em bases mais desafiadoras -- embora neste trabalho isso não tenha sido necessário, dada a qualidade e o alinhamento já presentes nas imagens utilizadas. Outro caminho promissor seria testar a abordagem em cenários mais realistas, com expressões faciais variadas, como caretas, sorrisos ou estados emocionais distintos, para avaliar a capacidade das \textit{splines} em representar feições mais dinâmicas. Também é válido considerar uma análise mais aprofundada sobre os hiperparâmetros da etapa de detecção de bordas, como os limiares do algoritmo de Canny, e seu impacto na acurácia final do sistema. Por fim, etapas de pós-processamento, como a adição controlada de ruídos ou perturbações nas imagens, poderiam ser utilizadas para testar a resiliência do modelo frente a distorções comuns em aplicações práticas, contribuindo para uma avaliação mais abrangente da robustez da abordagem proposta.