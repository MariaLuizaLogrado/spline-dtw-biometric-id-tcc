\chapter{RESULTADOS} \label{cha:resultados}

Para a avaliação dos métodos propostos neste trabalho, foi utilizada uma amostra do banco de dados FERET \cite{FERET1,FERET2}. A seleção compreende um total de 24 imagens faciais pertencentes a quatro indivíduos distintos. Ressalta-se que há um desbalanceamento na distribuição de imagens por indivíduo, sendo 10 imagens do primeiro sujeito, 3 do segundo, 6 do terceiro e 5 do quarto. Todas as imagens foram capturadas em condições controladas, com orientação frontal e sem variações significativas de ângulo ou expressões faciais.

Foi preciso segmentar quatro regiões distintas: olhos, nariz e boca, cada uma representada por um array bidimensional contendo as coordenadas dos pontos relevantes do contorno contendo as \textit{splines}.

O DTW foi aplicado separadamente para cada uma dessas quatro representações, permitindo avaliar a similaridade local (por região) e utilizando o sistema de votação para determinar a identidade do indivíduo. Além disso, uma outra versão com todas as características concatenadas foi utilizada como referência global da face.

É importante ressaltar que devido a uma aleatoriedade introduzida de propósito no algoritmo para retirar alguns pontos (\autoref{sec:resultado-agm}), os resultados podem variar entre as execuções. Portanto, para garantir a consistência dos resultados, foi utilizada a semente \texttt{random.seed(42)} em todas as execuções, assegurando que os mesmos pontos fossem selecionados em cada teste.

A máquina utilizada para os testes possui as seguintes especificações: processador Intel Core i7-8550U, 16 GB de memória RAM, 4 núcleos e 8 processadores lógicos.

\section{Parâmetros de Configuração}
\label{sec:parametros-configuracao}

Para a realização dos experimentos, foram definidos alguns parâmetros de configuração que influenciam diretamente o desempenho do DTW e a qualidade das curvas geradas pelas \textit{splines}. Esses parâmetros incluem:
\begin{itemize}
    \item \textbf{Passo da \textit{Spline}:} Define a densidade dos pontos ao longo da curva. Um passo menor resulta em curvas mais detalhadas, mas com maior custo computacional.
    \item \textbf{Tensão:} Controla a suavidade das curvas geradas pelas \textit{splines}. Valores mais altos resultam em curvas mais flexíveis, enquanto valores mais baixos produzem curvas mais rígidas.
    \item \textbf{Translação:} Indica se as curvas devem ser transladadas para alinhar melhor as características faciais.
\end{itemize}

Para cada combinação desses parâmetros, foram realizados testes com as características faciais concatenadas e segmentadas, permitindo uma análise comparativa dos resultados obtidos.

\section{Resultados Obtidos}

 As tabelas \ref{tab:resultados-concatenados} e \ref{tab:resultados-segmentados} apresentam os resultados obtidos para cada uma das abordagens, considerando diferentes parâmetros de configuração. Esses parâmetros influenciam diretamente na suavidade, na flexibilidade e na posição das curvas geradas, impactando a acurácia dos resultados obtidos pelo DTW.

\begin{table}[h!]
    \centering
    \caption{Testes realizados para as características concatenadas.}
    \begin{tabular}{|c|c|c|c|c|c|c|c|}
    \hline
    \textbf{Experimento} & \textbf{Passo \textit{Spline}} & \textbf{Tensão} & \textbf{Translação} & \textbf{Acurácia (\%)} & \textbf{Tempo} \\
    \hline
    1 & 0.5 & 1 & Não & 50 & 4min 21s\\
    2 & 0.5 & 0.5 & Não & 54.16 & 4min 05s\\
    3 & 0.5 & 0.25 & Não & 50 & 4min 13s\\

    4 & 1 & 1 & Não & 50 & 2min 02s\\
    5 & 1 & 0.5 & Não & 54.16 & 1min 48s\\
    6 & 1 & 0.25 & Não & 45.83 & 2min 00s\\

    7 & 0.5 & 1 & Sim & 16.66 & 4min 33s \\
    8 & 0.5 & 0.5 & Sim & 41.66 & 3min 57s \\
    9 & 0.5 & 0.25 & Sim & 37.5 & 5min 05s\\
 
    10 & 1 & 1 & Sim & 33.33 & 1min 58s \\
    11 & 1 & 0.5 & Sim & 33.33 & 2min 10s \\
    12 & 1 & 0.25 & Sim & 33.33 & 2min 04s \\
    \hline
    \end{tabular}
    \label{tab:resultados-concatenados}
\end{table}

\begin{table}[h!]
    \centering
    \caption{Testes realizados para as características segmentadas.}
    \begin{tabular}{|c|c|c|c|c|c|c|c|}
    \hline
    \textbf{Experimento} & \textbf{Passo \textit{Spline}} & \textbf{Tensão} & \textbf{Translação} & \textbf{Acurácia (\%)} & \textbf{Tempo} \\
    \hline

    13 & 0.5 & 1 & Não & 41.66 & 2min 12s \\
    14 & 0.5 & 0.5 & Não & 41.66 & 2min 39s \\
    15 & 0.5 & 0.25 & Não & 41.66 & 2min 21s \\
    
    16 & 1 & 1 & Não & 45.83 & 1min 07s \\
    17 & 1 & 0.5 & Não & 37.5 & 1min 08s \\
    18 & 1 & 0.25 & Não & 41.66 & 1min 08s \\
 
    19 & - & - & Não & 37.5 & 0min 21s\\
 
    20 & 0.5 & 1 & Sim & 79.16 & 2min 43s \\
    21 & 0.5 & 0.5 & Sim & 75 & 2min 39s \\
    22 & 0.5 & 0.25 & Sim & 70.83 & 2min 27s \\
 
    23 & 1 & 1 & Sim & 75 & 1min 04s \\
    24 & 1 & 0.5 & Sim & 79.16 & 1min 09s \\
    25 & 1 & 0.25 & Sim & 79.16 & 1min 13s \\
 
    26 & - & - & Sim & 70.83 & 0min 17s\\
    \hline
    \end{tabular}
    \label{tab:resultados-segmentados}
\end{table}

\subsection{Características Concatenadas}

Na Tabela \ref{tab:resultados-concatenados}, observa-se que a introdução de translação causou uma queda significativa na acurácia. Enquanto os experimentos sem translação atingiram até 54,16\% de acurácia (Experimentos 2 e 5), aqueles com translação tiveram desempenho inferior, com valores que variaram entre 16,66\% e 41,66\%. Isso sugere que, ao utilizar características concatenadas, a translação dificulta a extração de padrões discriminativos.

Além disso, ao comparar os valores de tensão, não há uma tendência clara de melhora ou piora na acurácia. No entanto, o passo da \textit{spline} igual a 1 reduziu consideravelmente o tempo de execução, com destaque para o Experimento 5 (1min 48s), mantendo a mesma acurácia de 54,16\% que o Experimento 2 (4min 05s), cujo passo era 0.5. Isso indica que um passo maior pode ser preferível por reduzir o tempo computacional sem comprometer o desempenho.

\subsection{Características Segmentadas}

Na Tabela \ref{tab:resultados-segmentados}, o cenário se inverte: os melhores resultados foram obtidos com a presença de translação. Os experimentos 20, 24 e 25 atingiram 79,16\% de acurácia, superando significativamente os demais. Isso indica que, no caso das características segmentadas, a translação introduz variações benéficas ao aprendizado, provavelmente por aumentar a robustez das representações faciais.

Entre os experimentos sem translação, a acurácia ficou abaixo de 46\%, indicando que a segmentação sozinha não é suficiente para gerar boas representações sem esse tipo de transformação. Já o tempo de execução foi consistentemente mais baixo nas segmentadas em comparação às concatenadas, sugerindo uma vantagem adicional em termos de eficiência computacional.

Os experimentos 19 e 26 utilizaram apenas os pontos extraídos automaticamente, sem aplicação de \textit{splines}. Esses testes funcionam como uma linha de base para comparação com as demais abordagens. Observa-se que, ao comparar o Experimento 19 (sem translação) com o Experimento 26 (com translação), a acurácia salta de 37,5\% para 70,83\%, evidenciando o impacto positivo da translação mesmo quando se utiliza uma representação simples baseada apenas nos pontos faciais. Isso reforça a relevância da translação como uma etapa de pré-processamento importante, especialmente no contexto das características segmentadas.

\subsection{Considerações Finais}

De forma geral:

\begin{itemize}
    \item A translação impacta negativamente nas características concatenadas, mas melhora significativamente os resultados nas características segmentadas.
    \item Um passo da \textit{spline} maior (1) reduz o tempo computacional sem grandes perdas de desempenho.
    \item As características segmentadas com translação se mostraram a melhor combinação, atingindo a maior acurácia com tempos de execução relativamente baixos.
\end{itemize}

Esses resultados indicam que a segmentação, aliada a transformações geométricas simples como a translação, pode ser uma estratégia promissora para a extração de características mais robustas em sistemas biométricos baseados em curvas \textit{spline}.