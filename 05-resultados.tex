\chapter{RESULTADOS} \label{cha:resultados}

Para a avaliação dos métodos propostos neste trabalho, foi utilizada uma amostra do banco de dados FERET \cite{FERET1,FERET2}. A seleção consistiu em um total de 24 imagens, pertencentes a quatro indivíduos distintos. É importante ressaltar que há um desbalanceamento na quantidade de imagens por indivíduo e todas as imagens foram capturadas sob condições controladas em orientação frontal, sem grandes variações de ângulo ou expressões faciais. %detalhar imagens por indivíduo

Foi preciso segmentar quatro regiões distintas: olhos, nariz e boca, cada uma representada por um array bidimensional contendo as coordenadas dos pontos relevantes do contorno contendo as \textit{splines}.

O DTW foi aplicado separadamente para cada uma dessas quatro representações, permitindo avaliar a similaridade local (por região) e utilizando o sistema de votação para determinar a identidade do indivíduo. Além disso, uma outra versão com todas as características concatenadas foi utilizada como referência global da face.

É importante ressaltar que devido a uma aleatoriedade introduzida de propósito no algoritmo para retirar alguns pontos (\autoref{sec:resultado-agm}), os resultados podem variar entre as execuções. Portanto, para garantir a consistência dos resultados, foi utilizada a semente \texttt{random.seed(42)} em todas as execuções, assegurando que os mesmos pontos fossem selecionados em cada teste.

A máquina utilizada para os testes possui as seguintes especificações: processador Intel Core i7-8550U, 16 GB de memória RAM, 4 núcleos e 8 processadores lógicos.

A seguir, estão apresentadas as tabelas com os resultados obtidos para cada uma das abordagens, considerando diferentes parâmetros de configuração, como o passo do spline, a translação e a tensão. Esses parâmetros influenciam diretamente na suavidade e na flexibilidade das curvas geradas, impactando a acurácia dos resultados obtidos pelo DTW.

\begin{table}[h!]
    \centering
    \caption{Testes realizados para as características concatenadas.}
    \begin{tabular}{|c|c|c|c|c|c|c|}
    \hline
    \textbf{Passo Spline} & \textbf{Translação} & \textbf{Tensão} & \textbf{Acurácia (\%)} & \textbf{Tempo} \\
    \hline
    0.5 & Não & 1 & 50 & 4min 21s\\
    0.5 & Não & 0.5 & 54.16 & 4min 05s\\
    0.5 & Não & 0.25 & 50 & 4min 13s\\

    1 & Não & 1 & 50 & 2min 02s\\
    1 & Não & 0.5 & 54.16 & 1min 48s\\
    1 & Não & 0.25 & 45.83 & 2min 00s\\

    0.5 & Sim & 1 & 16.66 & 4min 33s \\
    0.5 & Sim & 0.5 & 41.66 & 3min 57s \\
    0.5 & Sim & 0.25 & 37.5 & 5min 05s\\

    1 & Sim & 1 & 33.33 & 1min 58s \\
    1 & Sim & 0.5 & 33.33 & 2min 10s \\
    1 & Sim & 0.25 & 33.33 & 2min 04s \\
    \hline
    \end{tabular}
\end{table}

\begin{table}[h!]
    \centering
    \caption{Testes realizados para as características segmentadas.}
    \begin{tabular}{|c|c|c|c|c|c|c|}
    \hline
    \textbf{Passo Spline} & \textbf{Translação} & \textbf{Tensão} & \textbf{Acurácia (\%)} & \textbf{Tempo} \\
    \hline

    0.5 & Não & 1 & 41.66 & 2min 12s \\
    0.5 & Não & 0.5 & 41.66 & 2min 39s \\
    0.5 & Não & 0.25 & 41.66 & 2min 21s \\

    1 & Não & 1 & 45.83 & 1min 07s \\
    1 & Não & 0.5 & 37.5 & 1min 08s \\
    1 & Não & 0.25 & 41.66 & 1min 08s \\

    - & Não & - & 37.5 & 0min 21s\\

    0.5 & Sim & 1 & 79.16 & 2min 43s \\
    0.5 & Sim & 0.5 & 75 & 2min 39s \\
    0.5 & Sim & 0.25 & 70.83 & 2min 27s \\
 
    1 & Sim & 1 & 75 & 1min 04s \\
    1 & Sim & 0.5 & 79.16 & 1min 09s \\
    1 & Sim & 0.25 & 79.16 & 1min 13s \\

    - & Sim & - & 70.83 & 0min 17s\\
    \hline
    \end{tabular}
\end{table}
