%%%%%%%%%%%%%%%%%%%%%%%%%%%%%%%%%%%%%%%%%%%%%%%%%%%%%%%
% Arquivo para entrada de dados para a parte pré textual
%%%%%%%%%%%%%%%%%%%%%%%%%%%%%%%%%%%%%%%%%%%%%%%%%%%%%%%
% 
% Basta digitar as informações indicidas, no formato 
% apresentado.
%
%%%%%%%
% Os dados solicitados são, na ordem:
%
% tipo do trabalho
% componentes do trabalho 
% título do trabalho
% nome do autor
% local 
% data (ano com 4 dígitos)
% orientador(a)
% coorientador(a)(as)(es)
% arquivo com dados bibliográficos
% instituição
% setor
% programa de pós gradução
% curso
% preambulo
% data defesa
% CDU
% errata
% assinaturas - termo de aprovação
% resumos & palavras chave
% agradecimentos
% dedicatoria
% epígrafe


% Informações de dados para CAPA e FOLHA DE ROSTO
%----------------------------------------------------------------------------- 
\tipotrabalho{Trabalho Acadêmico}
%    {Relatório Técnico}
%    {Dissertação}
%    {Tese}
%    {Monografia}

% Marcar Sim para as partes que irão compor o documento pdf
%----------------------------------------------------------------------------- 
 \providecommand{\terCapa}{Sim}
 \providecommand{\terFolhaRosto}{Sim}
 \providecommand{\terTermoAprovacao}{Nao}
 \providecommand{\terDedicatoria}{Nao}
 \providecommand{\terFichaCatalografica}{Nao}
 \providecommand{\terEpigrafe}{Nao}
 \providecommand{\terAgradecimentos}{Nao}
 \providecommand{\terErrata}{Nao}
 \providecommand{\terListaFiguras}{Nao}
 \providecommand{\terListaQuadros}{Nao}
 \providecommand{\terListaTabelas}{Nao}
 \providecommand{\terSiglasAbrev}{Nao} 
 \providecommand{\terSimbolos}{Nao}
 \providecommand{\terResumos}{Sim}            
 \providecommand{\terSumario}{Sim}
 \providecommand{\terAnexo}{Nao}
 \providecommand{\terApendice}{Nao}
 \providecommand{\terIndiceR}{Nao}
%----------------------------------------------------------------------------- 

% \titulo{\textit{Splines} para Representação de Características Faciais com Aplicação em Biometria}
\titulo{Modelagem de Feições Faciais com \textit{Splines} para Reconhecimento por \textit{Dynamic Time Warping}}
% \titulo{Análise e Reconhecimento de Formas Faciais com \textit{Splines} e Alinhamento Temporal Dinâmico}
\autor{Maria Luiza Sampaio Logrado}
\local{Curitiba}
\data{2025} %Apenas ano 4 dígitos

% Orientador ou Orientadora
\orientador{
Prof Thiago de Oliveira Quinelato, DSc.}

% ----------------------------------------------------------
\addbibresource{referencias.bib}

% ----------------------------------------------------------
\instituicao{%
Universidade Federal do Paraná}

\preambulo{
Trabalho apresentado como requisito parcial à coclusão do curso de graduação em Matemática Industrial no Setor de Exatas da Universidade Federal do Paraná}

%----------------------------------------------------------------------------- 

\newcommand{\imprimirCurso}{}
%Programa de P\'os Gradua\c{c}\~ao em Engenharia da Constru\c{c}\~ao Civil}

\newcommand{\imprimirDataDefesa}{
09 de Dezembro de 2018}

\newcommand{\imprimircdu}{
02:141:005.7}

% ----------------------------------------------------------
\newcommand{\imprimirerrata}{
Elemento opcional da \cites[4.2.1.2]{NBR14724:2011}. Exemplo:

\vspace{\onelineskip}

FERRIGNO, C. R. A. \textbf{Tratamento de neoplasias ósseas apendiculares com
reimplantação de enxerto ósseo autólogo autoclavado associado ao plasma
rico em plaquetas}: estudo crítico na cirurgia de preservação de membro em
cães. 2011. 128 f. Tese (Livre-Docência) - Faculdade de Medicina Veterinária e
Zootecnia, Universidade de São Paulo, São Paulo, 2011.

\begin{table}[htb]
\center
\footnotesize
\begin{tabular}{|p{1.4cm}|p{1cm}|p{3cm}|p{3cm}|}
  \hline
   \textbf{Folha} & \textbf{Linha}  & \textbf{Onde se lê}  & \textbf{Leia-se}  \\
    \hline
    1 & 10 & auto-conclavo & autoconclavo\\
   \hline
\end{tabular}
\end{table}}

% Comandos de dados - Data da apresentação
\providecommand{\imprimirdataapresentacaoRotulo}{}
\providecommand{\imprimirdataapresentacao}{}
\newcommand{\dataapresentacao}[2][\dataapresentacaoname]{\renewcommand{\dataapresentacao}{#2}}

% Comandos de dados - Nome do Curso
\providecommand{\imprimirnomedocursoRotulo}{}
\providecommand{\imprimirnomedocurso}{}
\newcommand{\nomedocurso}[2][\nomedocursoname]
  {\renewcommand{\imprimirnomedocursoRotulo}{#1}
\renewcommand{\imprimirnomedocurso}{#2}}


% ----------------------------------------------------------
\newcommand{\AssinaAprovacao}{

\assinatura{%\textbf
   {Professora} \\ UFPR}
   \assinatura{%\textbf
   {Professora} \\ ENSEADE}
   \assinatura{%\textbf
   {Professora} \\ TIT}
   %\assinatura{%\textbf{Professor} \\ Convidado 4}
      
   \begin{center}
    \vspace*{0.5cm}
    %{\large\imprimirlocal}
    %\par
    %{\large\imprimirdata}
    \imprimirlocal, \imprimirDataDefesa.
    \vspace*{1cm}
  \end{center}
  }
  
% ----------------------------------------------------------
%\newcommand{\Errata}{%\color{blue}
%Elemento opcional da \textcite[4.2.1.2]{NBR14724:2011}. Exemplo:
%}

% ----------------------------------------------------------
\newcommand{\EpigrafeTexto}{%\color{blue}
\textit{``Não vos amoldeis às estruturas deste mundo, \\
		mas transformai-vos pela renovação da mente, \\
		a fim de distinguir qual é a vontade de Deus: \\
		o que é bom, o que Lhe é agradável, o que é perfeito.\\
		(Bíblia Sagrada, Romanos 12, 2)}
}

% ----------------------------------------------------------
\newcommand{\ResumoTexto}{%\color{blue}
Este trabalho investiga o uso de \textit{splines} cúbicas como método de representação de características faciais em sistemas biométricos. A proposta tem como objetivo avaliar se a modelagem das formas faciais por meio de curvas suaves pode contribuir para uma identificação mais eficiente e interpretável de indivíduos. O processo inicia-se com a extração automática de pontos a partir de imagens faciais, utilizando técnicas como detecção com Haar Cascades, contornos via operador de Canny e redução de pontos com base em grafos e árvores geradoras mínimas. Em seguida, as curvas obtidas são modeladas com \textit{splines} cúbicas, que fornecem uma representação contínua e compacta das feições. Para realizar a comparação entre diferentes indivíduos, emprega-se o algoritmo \textit{Dynamic Time Warping}, capaz de alinhar sequências mesmo com variações espaciais. Os resultados mostram que a abordagem é viável e apresenta bom desempenho na tarefa de reconhecimento facial, destacando o potencial de combinar métodos geométricos com algoritmos clássicos de comparação temporal. O trabalho também aponta caminhos para futuras melhorias, como o uso de expressões faciais variadas, ajustes nos hiperparâmetros de pré-processamento e testes com bases de dados mais desafiadoras.
} 

\newcommand{\PalavraschaveTexto}{%\color{blue}
reconhecimento facial, \textit{splines} cúbicas, DTW, biometria.}

% ----------------------------------------------------------
\newcommand{\AbstractTexto}{%\color{blue}
This work investigates the use of cubic splines as a method for representing facial features in biometric systems. The aim is to assess whether modeling facial shapes using smooth curves can contribute to more efficient and interpretable individual identification. The process begins with the automatic extraction of key points from facial images, using techniques such as Haar Cascade detection, contour extraction via the Canny operator, and point reduction based on graph structures and minimum spanning trees. Next, the resulting curves are modeled with cubic splines, which provide a continuous and compact representation of facial features. To compare different individuals, the Dynamic Time Warping algorithm is employed, as it can align sequences even in the presence of spatial variations. The results show that the proposed approach is feasible and performs well in the task of facial recognition, highlighting the potential of combining geometric methods with classical temporal comparison algorithms. The work also outlines paths for future improvements, such as incorporating varied facial expressions, adjusting preprocessing hyperparameters, and testing with more challenging datasets.
}
% ---
\newcommand{\KeywordsTexto}{%\color{blue}
facial recognition, cubic splines, DTW, biometrics
}

% ----------------------------------------------------------
\newcommand{\Resume}
{%\color{blue}
%Il s'agit d'un résumé en français.
} 
% ---
\newcommand{\Motscles}
{%\color{blue}
 %latex. abntex. publication de textes.
}

% ----------------------------------------------------------
\newcommand{\Resumen}
{%\color{blue}
%Este es el resumen en español.
}
% ---
\newcommand{\Palabrasclave}
{%\color{blue}
%latex. abntex. publicación de textos.
}

% ----------------------------------------------------------
\newcommand{\AgradecimentosTexto}{%\color{blue}
Os agradecimentos principais são direcionados à Gerald Weber, Miguel Frasson, Leslie H. Watter, Bruno Parente Lima, Flávio de  Vasconcellos Corrêa, Otavio Real Salvador, Renato Machnievscz\footnote{Os nomes dos integrantes do primeiro
projeto abn\TeX\ foram extraídos de \url{http://codigolivre.org.br/projects/abntex/}} e todos aqueles que contribuíram para que a produção de trabalhos acadêmicos conforme as normas ABNT com \LaTeX\ fosse possível.

Agradecimentos especiais são direcionados ao Centro de Pesquisa em Arquitetura da Informação\footnote{\url{http://www.cpai.unb.br/}} da Universidade de Brasília (CPAI), ao grupo de usuários
\emph{latex-br}\footnote{\url{http://groups.google.com/group/latex-br}} e aos novos voluntários do grupo \emph{\abnTeX}\footnote{\url{http://groups.google.com/group/abntex2} e
\url{http://abntex2.googlecode.com/}}~que contribuíram e que ainda
contribuirão para a evolução do \abnTeX.

Os agradecimentos principais são direcionados à Gerald Weber, Miguel Frasson, Leslie H. Watter, Bruno Parente Lima, Flávio de Vasconcellos Corrêa, Otavio Real Salvador, Renato Machnievscz\footnote{Os nomes dos integrantes do primeiro
projeto abn\TeX\ foram extraídos de \url{http://codigolivre.org.br/projects/abntex/}} e todos aqueles que contribuíram para que a produção de trabalhos acadêmicos conforme as normas ABNT com \LaTeX\ fosse possível.
}

% ----------------------------------------------------------
\newcommand{\DedicatoriaTexto}{%\color{blue}
\textit{ Dedico este trabalho a meus pais e seus respectivos cônjuges, \\
        que sempre me apoiaram e incentivaram durante essa jornada.}
	}

